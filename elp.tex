\documentclass{llncs}

\usepackage[utf8]{inputenc}
\usepackage[T1]{fontenc}

\usepackage{hyperref}

% Needs splncs03.bst via Springer
\bibliographystyle{splncs03}

\usepackage{amsmath, amssymb} % For \flalign, \top, ...
\usepackage[ruled,linesnumbered]{algorithm2e}

\usepackage[table,usenames,dvipsnames]{xcolor}
\usepackage{nicefrac,xspace,csquotes}

\newcommand{\alc}{\ensuremath{\mathcal{ALC}}\xspace}
\newcommand{\al}{\ensuremath{\mathcal{AL}}\xspace}
\newcommand{\elp}{\ensuremath{\mathcal{EL^+}}\xspace}
\newcommand{\el}{\ensuremath{\mathcal{EL}}\xspace}
\newcommand{\hl}{\ensuremath{\mathcal{HL}}\xspace}
\newcommand{\tb}{\ensuremath{\mathcal{T}}\xspace} % TBox
\newcommand{\ab}{\ensuremath{\mathcal{A}}\xspace} % ABox
\newcommand{\nc}{\ensuremath{\mathsf{N_C}}\xspace} % concept names
\newcommand{\nr}{\ensuremath{\mathsf{N_R}}\xspace} % role names

\newcommand{\snomed}{\textsc{Snomed CT}\xspace}
\newcommand{\galen}{\textsc{Galen}\xspace}

\newcommand{\subsume}{\sqsubseteq}

\title{Pinpointing in the Description Logic \elp}
\author{Lorenz~Leutgeb}
\institute{International Center for Computational Logic, TU Dresden\\ \email{lorenz.leutgeb@mailbox.tu-dresden.de}}

\begin{document}

\maketitle

\begin{abstract}
Description Logic (DL) is an area of active research in the field knowledge representation and artificial intelligence. It is concerned with representing concepts in formal logics whose expressiveness lies between propositional logic and first order logic, and (computationally) reasoning about them.
A key reasoning problem is the question of whether one concept is subsumed by another. Further, given such a subsumption relationship, the goal of axiom pinpointing is to find those axioms that give rise to the subsumption.
We are concerned with the the description logic \elp in particular, for which subsumption can be solved in polynomial time. An algorithm for axiom pinpointing, based on a subsumption algorithm, is discussed
in detail. We briefly summarize complexity results and mention weaker variants that still behave well in practice.
\end{abstract}

\section{Introduction}
\label{intro}

Description Logic (DL)\marginpar{DL (field)} is a successful subfield of research in Knowledge Representation and Reasoning (KRR). At its core lies a family of logic languages, called Description Logics \marginpar{DLs (logics)} (DLs). Most of these logics lie in decidable fragments of classical first order logic. Analysis of relations between expressiveness of logics and tractability of reasoning problems defined over them is a big effort within DL.

From an application standpoint\marginpar{Applications}, DLs are used to structure conceptual knowledge in problem domains like natural language processing, databases and ontologies. The Web Ontology Language (OWL), fundamental to the semantic web, is formally supported by Description Logic. Also, large biomedical knowledge bases like \snomed and \galen are expressed in DLs.

A common reasoning problem is \emph{concept subsumption}\marginpar{Motivation}.\footnote{Other reasoning problems include \emph{satisfiability} and \emph{equivalence} of concepts, as well as \emph{consistency} of knowledge bases. Concept subsumption is the most relevant for this work.} Intuitively, this is the same as asking whether all objects that belong to concept $C$ also belong to concept $D$, under the axioms of a given knowledge base, i.e.~$D$ \emph{subsumes} $C$ wrt. the axioms. With the advent of large knowledge bases (hundreds of thousands axioms), new challenges arise: Assume that under the axioms of a large knowledge base, $D$ subsumes $C$, but it turns out that this consequence is not consistent with observations in the real world, so the knowledge base must be adjusted to better model reality. Now, which of the thousands of axioms give rise to the faulty conclusion? We call this problem \emph{axiom pinpointing}.

Naturally, we are interested in \textbf{computing \underline{min}imal \underline{a}xiom sets (MinA)} that explain why $D$ subsumes $C$.

In this work\marginpar{Contents} we summarize results on axiom pinpointing for the description logic \elp. First, in Section \ref{sec:prelim}, we revisit syntax and semantics of the logics under consideration. Section \ref{sec:algos} is meant to showcase different approaches to axiom pinpointing. We follow the distinction made in \cite[Section 1]{family}: A \emph{glass-box} algorithm based on a polynomial-time algorithm for subsumption in \elp is discussed in \ref{sec:glass}, while in \ref{sec:black} we address two \emph{black-box} algorithms. We defer all complexity results to Section \ref{sec:cmplx} where we address the complexity of finding a single (minimal) set of axioms (Section \ref{sec:one}) separately from the complexity of finding all (minimal) sets of axioms (Section \ref{sec:all}).

Note that our investigation starts with the work by Baader, Pe{\~{n}}aloza and Suntisrivaraporn \cite{orig1,orig2}. We take into account subsequent results achieved by Pe{\~{n}}aloza and Sertkaya \cite{hard,family}.

\section{Preliminaries}
\label{sec:prelim}

In this section we introduce syntax and semantics of \elp as well as the core reasoning problem of concept subsumption.

Description Logic considers syntax called \emph{concept descriptions}\marginpar{Syntax}, inductively defined via \emph{constructors} from a set of concept names and a set of role names. The constructors of the respective languages relevant to this work are listed in the upper section of the column titled \emph{Syntax} in Table \ref{tbl:langs}.

\begin{table}
\caption{Overview of syntax and semantics of selected logics. Here, $A, B \in \nc$ are concept names, $C, D$ are concept descriptions (inductively), $r_1, \ldots, r_n, s \in \nr$ are role names. We denote an interpretation by $I = (\Delta^I, \cdot^I)$. GCI stands for General Concept Inclusion.}
\begin{center}
\rowcolors{2}{orange!5}{white}
\begin{tabular}{|l|cc|cccc|}
\hline
Name & Syntax & Semantics &~\al ~&~\el ~&~\elp ~&~\hl ~\\
\hline
\hline
Top & $\top$ & $\Delta^I$ & \checkmark & \checkmark & \checkmark & \checkmark \\
Bottom & $\bot$ & $\emptyset$ & \checkmark & & & \\
Atomic Negation & $ \neg A $ & $\Delta^I \setminus A^I$ & \checkmark & & & \\
Conjunction & $ C \sqcap D $ & $C^I \cap D^I$ & \checkmark & \checkmark & \checkmark & \checkmark \\
%yDisjunction & $ C \sqcup D $ & $C^I \cup D^I$ & \checkmark & & & \\
Value Restriction & $ \forall r . C $ & \shortstack[c]{$\{x \in \Delta^I \mid \forall y \in \Delta^I :$ \\[0mm] $(x, y) \in r^I \rightarrow y \in C^I \}$} & \checkmark & & & \\
Existential Restr.~& $ \exists r . C $ & \shortstack[c]{$ \{x \in \Delta^I \mid \exists y \in \Delta^I :$ \\ $(x, y) \in r^I \wedge y \in C^I \} $} & & \checkmark & \checkmark & \\
\hline
\hline
GCI & $C \sqsubseteq D$ & $C^I \subseteq D^I$ & \checkmark & \checkmark & \checkmark & \checkmark \\
Concept Definition & $C \equiv D$ & $C^I = D^I$ & \checkmark & \checkmark & \checkmark & \checkmark \\
Role Inclusion & $r_1 \circ \cdots \circ r_n \sqsubseteq s$ & $r_1^I \circ \cdots \circ r_n^I \subseteq s^I$ & & & \checkmark & \\
\hline
\end{tabular}
\end{center}
\label{tbl:langs}
\end{table}

Semantics\marginpar{Semantics} are defined in terms of an interpretation $I$ which talks about a non-empty set of individuals denoted $\Delta^I$ by means of an interpretation function $\cdot^I$ as follows: $\cdot^I$ maps each concept name $A \in \nc$ to a set of individuals $A^I \subseteq \Delta^I$, and each role name $r \in \nr$ to a binary relation $r^I \subseteq \Delta^I \times \Delta^I$. Concept descriptions are interpreted by extending $\cdot^I$ as defined in the upper section of the column titled \emph{Semantics} in Table \ref{tbl:langs}.

In our setting, knowledge bases take the form of finite sets of axioms. We call such a set \emph{TBox}\marginpar{TBox}, denoted by $\tb$. Axioms are relations between concept descriptions, and their form is shown in the lower section of Table \ref{tbl:langs}. We say that an axiom is \emph{satisfied} under $I$ if the corresponding condition (column titled \emph{Semantics}) holds. Further, we say that an interpretation $I$ \emph{satisfies} $\tb$ if all axioms in $\tb$ are satisfied by $I$.

Since all \elp TBoxes are consistent by definition (there are no constructors that may cause logical inconsistencies \cite[Sec.~2]{orig1,orig2}), satisfiability of concepts and the consistency of TBoxes are trivial. Instead, we are generally interested in deciding the reasoning problem of \emph{concept subsumption}.

\begin{definition}[{Concept Subsumption, cf.~\cite[Def.~1]{orig1}}]
\label{def:subs}
Given \marginpar{$C \subsume_{\tb} D$} two \elp concept descriptions $C, D$ and an $\elp$ TBox \tb, $C$ is subsumed by $D$ wrt.~\tb (written $C \sqsubseteq_\tb D$) if for every interpretation $I$ that satisfies \tb we have that $C^I \subseteq D^I$.
\end{definition}

With this, we proceed to define minimal axiom sets which are crucial to this work.

\begin{definition}[{MinA, cf.~\cite[Def.~2 without partitioning $\tb$]{orig1}}]
\label{def:mina}
Let \marginpar{MinA} $\tb$ be an \elp TBox and $A, B$ concept names occurring in it such that $A \subsume_\tb B$. Then a \emph{minimal axiom set (MinA)} for \tb wr.t.~$A \subsume B$ is a subset $\mathcal{S} \subseteq \tb$ such that $A \subsume_\mathcal{S} B$ but $A \not \subsume_\mathcal{S'} B$ for all strict subsets $\mathcal{S'} \subset S$.
\end{definition}

\begin{example}[{Humans and Animals (cf.~\cite[Sec.~3]{orig1,orig2})}]
\label{ex:ha}
Consider the following TBox \tb containing axioms that model knowledge about species:
\begin{flalign}
\mathsf{Human} &\subsume \exists \mathsf{parent} . \mathsf{Human} & \text{Humans have a human parent.} \tag{$a_1$} \\
\mathsf{Human} &\subsume \mathsf{Monkey} & \text{Humans are monkeys.} \tag{$a_2$} \\
\exists \mathsf{parent} . \mathsf{Monkey} &\subsume \mathsf{Animal} & \text{Monkey parent? It's an animal.} \tag{$a_3$} \\
\mathsf{Monkey} &\subsume \mathsf{Animal} & \text{Monkeys are animals.} \tag{$a_4$} \\
\mathsf{Fish} &\subsume \mathsf{Animal} & \text{Fish are animals.} \tag{$a_5$}
\end{flalign}

Observe that $\mathsf{Human} \subsume_\tb \mathsf{Animal}$ holds. Now assume evidence has been found that not all humans are animals, thus we should adjust our TBox such that this relationship will not hold anymore. What are the axioms that give rise to the subsumption? We can see that two minimal sets of axioms explain it: $\mathcal{S}_1 = \{a_2, a_4\}$ and $\mathcal{S}_2 = \{ a_1, a_2, a_3 \}$. Even though $|\mathcal{S}_1| < |\mathcal{S}_2|$ both are MinAs, since we cannot find strict subsets that explain why all humans are animals.

Also note that the consequence holds when removing either $a_1$ and $a_3$ or $a_4$ alone and that $a_2$ is contained in both sets. The axiom $a_5$ is not relevant in our scenario, and we may imagine that there are thousands of other irrelevant axioms in \tb.

We hypothesize that $a_4$ is indeed a sensible axiom and $a_1, a_3$ are reasonable, but $a_2$ causes trouble since it states that humans are monkeys. If we remove $a_2$, then the resulting TBox will conform with our observation.
\end{example}

\section{Algorithms}
\label{sec:algos}

In this section we want to show how one MinA and multiple MinAs can be computed. We distinguish between \emph{glass-box} algorithms and \emph{black-box} algorithms. The former captures algorithms that may take advantage of the internals of the reasoning algorithm by inspecting axioms and their structure, generally using more involved data structures. On the contrary, the latter means algorithms that are built on top of basic reasoning services, thus are more abstract but can leverage already existing implementations of these reasoning services.

We will detail one glass-box approach and a black-box approach with some variations.

\subsection{Glass-Box Approach}
\label{sec:glass}

The glass box algorithm we present is designed as an extension of a polynomial-time algorithm for subsumption \cite{whatelse} in \elp (cf.~Def.~\ref{def:subs}). Therefore we first consider the algorithm for subsumption (Algorithm \ref{alg:sub}), and only then extend it to compute MinAs (Algorithm \ref{alg:pin}).

Following two simplifications based on \marginpar{Normalization} normalization greatly reduce the number of cases that need to be covered in the algorithm \cite[Sec.~2]{orig1}: (1) It is sufficient to address subsumption for concept names, given the following reduction: $C \sqsubseteq_\tb D$ if $A \sqsubseteq_{\tb \cup \{A \sqsubseteq C, D \sqsubseteq B \}} B$ where $A,B$ are new concept names not occurring in $C,D,\tb$ (for a proof concerning \el cf.~\cite[Lemma~6.1,p.~142]{book}). (2) It is possible to translate a TBox $\tb$ into an equivalent one in normal form (meaning that all GCIs are of the forms as presented in Table \ref{tbl:rules}, cf.~\cite[Fig.~6.1, p.~143]{book}) in time $\mathcal{O}(|\tb|)$ (proven in \cite[Lemma 6.2, p.~143]{book}).

\begin{algorithm}
  \KwIn{An \elp TBox $\tb$ in normal form over $\nc$ and $\nr$, and $A, B \in \nc$.}
  \KwOut{\enquote{yes} if $A \subsume_\tb B$ holds, \enquote{no} otherwise.}
  $\mathcal{A} := \{(A, A) \mid A \in \nc \}$ \tcp{Account for $A \subsume_\tb A$}
  $\hspace{4.9mm} \cup~\{(A,\top) \mid A \in \nc \}$ \tcp{Account for $A \subsume_\tb \top$}
  \While{there is an $i$ s.t. $1 \leq i \leq 5$, $a_i \in \tb$ and $P_i \subseteq \mathcal{A}$ \tcp{see Table \ref{tbl:rules}}}{
    $\mathcal{A} := \mathcal{A} \cup \{ q_i \}$ \tcp{Axiom $a_i$ explains $q_i$}
  }
  \If{$(A, B) \in \mathcal{A}$}{
    \Return{\emph{\enquote{yes}}}
  }
  \Return{\emph{\enquote{no}}}
  \caption{\textsc{Subsumption}($\tb$, $A$, $B$)}
  \label{alg:sub}
\end{algorithm}

Note that lines 3 and 5 in Algorithm \ref{alg:sub} refer to Table \ref{tbl:rules}, and that the choice of $i$ in those lines is don't care nondeterministic.

\begin{table}
\caption{Completion rules for subsumption in \elp, adapted from \cite[Fig.~1]{orig1,orig2}, see also \cite[Fig.~5.2, p.~104]{meng-phd}. Rule $i$ is applicable if $a_i \in \tb$ and $P_i \subseteq \ab$. If rule $i$ is applied, then $q_i$ is added to $\ab$.}
\begin{center}
\rowcolors{2}{orange!5}{white}
\begin{tabular}{|c|rcll|l|}
\hline
{\bfseries Rule} & \multicolumn{4}{c|}{\bfseries Applicability} & \multicolumn{1}{c|}{\bfseries Result} \\
$i$
 & \multicolumn{3}{c}{$a_i$ (GCI)} & $P_i$ (set of assertions) & $q_i$ (assertion) \\
\hline\hline
1& $A_1 \sqcap \cdots \sqcap A_n$&$\sqsubseteq$&$B$ & $(X, A_1), \ldots, (X,A_n)$ & $(X,B)$ \\
2& $A$&$\sqsubseteq$&$\exists r.B$ & $(X,A)$ & $(X,r,B)$ \\
3& $\exists r.A$&$\sqsubseteq$&$B$ & $(X,r,Y),(Y,A)$ & $(X,B)$ \\
4& $r$&$\sqsubseteq$&$s$ & $(X,r,Y)$ & $(X,s,Y)$ \\
5& $r \circ r'$&$\sqsubseteq$&$s$ & $(X,r,Y),(Y,r',Z)$ & $(X,s,Z)$\\
\hline
\end{tabular}
\end{center}
\label{tbl:rules}
\end{table}

Now we extend Algorithm \ref{alg:sub} to compute all MinAs in two steps: First, we modify it so that a monotone Boolean \emph{pinpointing formula} that corresponds to the subsumption relation in question is generated via labeling of assertions. Second, we obtain MinAs by finding models for the pinpointing formula, such that the interpretation of propositional variables corresponds to a selection of axioms in the MinAs.

By means of a labeling function\marginpar{Labeling} $lab : \tb \longrightarrow \mathcal{R}$ (where $\mathcal{R}$ is a set of propositional variables) every axiom $a \in \tb$ is labeled with a unique propositional variable $lab(a) \in \mathcal{R}$, and we denote by $lab(\tb) \subseteq \mathcal{R}$ the set of labels of all axioms in $\tb$. A \emph{monotone Boolean formula over $lab(\tb)$} is a Boolean formula using (some of) the propositional variables $lab(\tb)$ and only the binary connectives conjunction and disjunction, as well as the nullary connective \texttt{true} for truth. A \marginpar{Valuation}\emph{valuation} is characterized by the set of propositional variables that it makes true. For a valuation $\mathcal{V} \subseteq lab(\tb)$ let $\tb_\mathcal{V} = \{ a \in \tb \mid lab(a) \in \mathcal{V}\}$ be the selection of axioms for which their corresponding label is made true under $\mathcal{V}$.

\begin{definition}[{Pinpointing Formula \cite[Def.~3]{orig1}}]
Given \marginpar{Pinpointing Formula} an \elp TBox \tb and concept names $A,B$ occuring in it, a monotone Boolean formula $\psi$ over $lab(\tb)$ is a \emph{pinpointing formula} for $\tb$ wrt.~$A \subsume B$ if for every valuation $\mathcal{V} \subseteq lab(\tb)$ it holds that $A \subsume_{\tb_\mathcal{V}} B$ iff $\mathcal{V}$ satisfies $\psi$.
\end{definition}

For the two MinAs in Example \ref{ex:ha}, $\mathcal{S}_1 = \{a_2, a_4\}$ and $\mathcal{S}_2 = \{ a_1, a_2, a_3 \}$, we obtain $a_2 \wedge (a_4 \vee (a_1 \wedge a_3))$ as a pinpointing formula.
Lastly, to establish a correspondence, we also need to address $\subseteq$-minimality of valuations.

\begin{proposition}[{\cite[Prop.~1]{orig1}}]
Let $\psi$ be a pinpointing formula for $\tb$ wrt.~$A \subsume B$. Then $$\big\{ \tb_\mathcal{V} \mid \mathcal{V} \text{ is a $\subseteq$-minimal valuation satisfying } \psi \big\}$$ is the set of all MinAs for $\tb$ wrt.~$A \subsume B$.
\end{proposition}

\begin{algorithm}
  \label{alg:allminas}
  \KwIn{An \elp TBox $\tb$ in normal form over $\nc$.}
  \KwOut{A set of assertions and a labeling function.}
  $\mathcal{A} := \{(A, A) \mid A \in \nc \}$ \label{alg:pin:acc:begin} \tcp{Account for $A \subsume_\tb A$.}
  $\hspace{4.6mm} \cup~\{(A,\top) \mid A \in \nc \}$ \label{alg:pin:acc:end} \tcp{Account for $A \subsume_\tb \top$.}
  $lab(a) := \texttt{true}$ for all $a \in \mathcal{A}$ \label{alg:pin:initlbl} \tcp{Initialize labels.}
  \While{there is an $i$ s.t. $1 \leq i \leq 5$, $a_i \in \tb$ and $P_i \subseteq \mathcal{A}$ \label{alg:pin:loop:begin}\tcp{see Table \ref{tbl:rules}}}{
      $\phi = lab(a_i) \wedge \bigwedge_{p \in P_i} lab(p)$ \\
      \uIf{$q_i \not \in \mathcal{A}$}{
          \tcp{$a_i$ is the only known explanation of $a_i$.}
        $\mathcal{A} := \mathcal{A} \cup \{ q_i \}$\\
        $lab(q_i) := \phi$\\
      }
      \Else{\label{alg:allminas:else:begin} \tcp{$q_i$ can already be explained, $a_i$ might be a new explanation!}
        $\psi = lab(q_i)$ \\
        \If{$\psi \vee \phi \not \equiv \psi$}{
          \tcp{Override, as $q_i$ might be explained by $\psi$ or $\phi$.}
          $lab(q_i) := \psi \vee \phi$
        }
        \label{alg:allminas:else:end}
      }
      \label{alg:pin:loop:end}
  }
  \Return{$(\mathcal{A}, lab)$}
  \caption{\textsc{AllMinAs}($\tb$)}
  \label{alg:pin}
\end{algorithm}

The initial assertions (added in lines \ref{alg:pin:acc:begin} and \ref{alg:pin:acc:end}) of the form $(A, A)$ and $(A, \top)$ where $A \in \nc$ are labeled with \texttt{true} in line \ref{alg:pin:initlbl}.

Note that Algorithm \ref{alg:pin} terminates in time exponential in $|\tb|$, and it allows us to find all MinAs not only for a specific subsumption relation $A \subsume_\tb B$, but we may query the returned set of assertions for $(A, B)$ and consider $lab((A, B))$ for any subsumption relation of interest.

More formally (cf.~\cite[Thm.~2]{orig1,orig2}), the resulting set of assertions $\mathcal{A}$, after termination of the loop in lines \ref{alg:pin:loop:begin}-\ref{alg:pin:loop:end} satisfies the following properties: (1) $A \subsume_\tb B$ if and only if $(A, B) \in \mathcal{A}$, and (2) $lab((A,B))$ is a pinpointing formula for $\tb$ wrt. $A \subsume B$.

As expected, this algorithm may be generalized so that the input need not be normalized (cf.~\cite[Sec.~3]{orig1,orig2} and \cite[Sec.~5.2, p.~107]{meng-phd}).

\subsection{Black-Box Approaches}
\label{sec:black}

In this section we want to present a na\"{i}ve algorithm and a slightly more involved improved variant, to compute \emph{one} minimal set of axioms for a given subsumption consequence, in a \emph{black-box} manner: Both Algorithm \ref{alg:linone} and \ref{alg:logone}, discussed below, only rely on deciding subsumption and counting/choosing elements of the input TBox. In contrast to Algorithm \ref{alg:sub} and its extension Algorithm \ref{alg:pin} from the previous section, the structure of individual axioms is not considered. Consequently, their high-level approach is exposed in a more direct fashion as we omit statements that deal with peculiarities, such as the structure of an axiom.

Algorithm \ref{alg:linone} linearily scans a working copy of the given knowledge base: For every axiom, it is tested whether the subsumption relation under question still holds without it. If this is the case, then it is removed from the working copy. Clearly, $\mathcal{O}(|\tb|)$ subsumption tests must be performed. The result is a minimal set of axioms.

\begin{algorithm}
  \KwIn{An \elp TBox $\tb = \{a_1, \ldots, a_n\}$ over $\nc$ and $\nr$, and $A, B \in \nc$.}
  \KwOut{If $A \subsume_\tb B$ holds, one MinA for $\tb$ wrt. $A \subsume B$, else $\emptyset$.}
  \If{$A \not \subsume_\tb B$}{
    \Return{$\emptyset$}
  }
  $\mathcal{S} := \tb$ \tcp{Initialize MinA to set of all axioms.}
  \For{$1 \leq i \leq n$ \label{alg:linone:loop:begin}}{
  \If{$A \subsume_{\mathcal{S} \setminus \{a_i\}} B$}{
    $\mathcal{S} := \mathcal{S} \setminus \{ a_i \}$ \tcp{Remove irrelevant axiom.}
  }
  \label{alg:linone:loop:end}
  }
  \Return{$\mathcal{S}$}
  \caption{\textsc{LinOneMinA}($\tb$, $A$, $B$)\newline (cf.~\cite[Alg.~1]{orig1}, \cite[Alg.~1]{debugging}, \cite[Alg.~7, p.~97]{meng-phd})}
  \label{alg:linone}
\end{algorithm}

Algorithm \ref{alg:logone} employs a divide-and-conquer approach reminiscent of binary search. The minimal axiom set $\mathcal{S}$ is constructed recursively. For every recursive call, the input TBox is split into two halves (lines \ref{alg:logone:split1} and \ref{alg:logone:split2}) and in case any of the two halves preserves $A \subsume B$, search is continued in this part of the input. Otherwise, $\mathcal{S}$ is constructed as the union of the result for both halves respectively. With this approach, only $\mathcal{O}((|\mathcal{S}| - 1) + |\mathcal{S}| \cdot \log(|\tb|/|\mathcal{S}|))$ subsumption tests need to be performed \cite[p.~89]{meng-phd}.

\begin{algorithm}
  \KwIn{An \elp TBox $\tb = \{a_1, \ldots, a_n\}$ over $\nc$ and $\nr$, and $A, B \in \nc$ s.t.~$A \subsume_{\tb} B$ holds, as well as a \enquote{support set} (of axioms) for $A \subsume_{\tb} B$.}
  \KwOut{minimal subset $\mathcal{S}' \subseteq \tb$ s.t.~$A \subsume_{\mathcal{S} \cup \mathcal{S'}} B$ holds.}
  \If{$|\tb| = 1$}{
    \Return{$\tb$}
  }
  $L := \{ a_i \in \tb \mid 1 \leq i \leq \lfloor \nicefrac{n}{2} \rfloor \}$ \label{alg:logone:split1}\\
  $R := \{ a_i \in \tb \mid \lfloor \nicefrac{n}{2} \rfloor + 1 \leq i \leq n \}$ \label{alg:logone:split2}\\
  \If{$A \subsume_{\mathcal{S} \cup L} B$}{
    \Return{\textsc{LogOneMinA}($L$, $A$, $B$, $\mathcal{S}$)}
  }
  \If{$A \subsume_{\mathcal{S} \cup R} B$}{
    \Return{\textsc{LogOneMinA}($R$, $A$, $B$, $\mathcal{S}$)}
  }
  $\mathcal{S}_1 := $ \textsc{LogOneMinA}($L$, $A$, $B$, $\mathcal{S} \cup R$)\\
  $\mathcal{S}_2 := $ \textsc{LogOneMinA}($R$, $A$, $B$, $\mathcal{S} \cup \mathcal{S}_1$)\\
  \Return{$\mathcal{S}_1 \cup \mathcal{S}_2$}
  \caption{\textsc{LogOneMinA}($\tb$, $A$, $B$, $\mathcal{S}$)\newline (cf.~\cite[Alg.~2]{debugging}, \cite[Alg.~8, p.~98]{meng-phd})}
  \label{alg:logone}
\end{algorithm}

\section{Complexity}
\label{sec:cmplx}

In this section we briefly address the computational complexity of computing all MinAs and one MinA respectively.

\subsection{All MinAs}
\label{sec:all}

Since there may be exponentially many MinAs for a given \elp TBox, it is clear that an exponential run-time in the worst-case cannot be avoided.

\begin{example}[{\cite[Ex.~1]{orig1,orig2}}]
Consider the language \hl, which is a fragment of \elp (cf.~Table \ref{tbl:langs}). Given $n \geq 1$, we construct a \hl TBox of size linear in $n$, that gives rise to $2^n$ MinAs: $$\tb_n := \big\{~~B_{i-1} \subsume P_i \sqcap Q_i,~~P_i \subsume B_i,~~Q_i \subsume B_i~~\mid~~1 \geq i \geq n~~\big\}$$ Note that there are $2^n$ MinAs for $\tb_n$ wrt.~$B_0 \subsume B_n$, since for each $i$ with $1 \leq i \leq n$ either the axiom $P_i \subsume B_i$ or the axiom $Q_i \subsume B_i$ alone might be in the MinA.
\end{example}

Further, it can be shown that computing any interesting property of the set of all MinAs is not possible in polynomial time (unless P = NP).

\begin{theorem}[{\cite[Thm.~3 without partitioning $\tb$]{orig1,orig2}}]
Given an \hl TBox $\tb$, concept names $A, B$ occurring in $\tb$, and a natural number $n$, it is NP-complete to decide whether or not there is a MinA $\mathcal{S}$ for $\tb$ wrt.~$A \subsume B$ such that $|\mathcal{S}| \leq n$.
\end{theorem}

With this, \cite{orig1,orig2} starts further investigation. In particular, since one might regard the problem of computing all MinAs as an enumeration problem, this raises the question whether there is an \emph{output polynomial} algorithm for it. An algorithm with this property would only take polynomial time in cases where the number of MinAs is polynomial in the size of the input. However, after showing that there is no output polynomial algorithm for computing all minimal satisfying valuations of monotone Boolean formulae (unless P = NP) \cite[Thm.~4]{orig1,orig2}, a negative result is presented:

\begin{theorem}[{\cite[Thm.~5]{orig1,orig2}}]
\label{thm:nopol}
There is no output polynomial algorithm that computes for a given \hl TBox $\tb = (\mathcal{T}_s \uplus \mathcal{T}_r)$ and concept names $A, B$ occurring in $\tb$, all MinAs for $\tb$ wrt.~$A \subsume B$, unless P = NP.
\end{theorem}

Note that the setting considered in \cite{orig1,orig2} and in the above theorem the TBox $\tb$ is split into \enquote{static} part of $\tb$, denoted $\tb_s$ and $\tb_r$, the \enquote{refutable} part respectively. These parts are disjoint as $\uplus$ stands for the \emph{disjoint union} of two sets.

The question whether Theorem \ref{thm:nopol} also holds for the case where $\tb_s = \emptyset$, which corresponds to the view in this work, is addressed in \cite[Thm.~2]{hard}, where the problem of computing all MinAs is shown to be \enquote{at least as hard as recognizing the set of all minimal transversals of a given hypergraph \cite{hypmintrav}, which is a prominent open problem in hypergraph theory} (quoting from \cite{hard}, importance of the problem surveyed in~\cite{hyperai}).

A comprehensive analysis of the complexity of computing all MinAs in different variants can be found in \cite{family}, where similar logics are considered.

\subsection{One MinA}
\label{sec:one}

Algorithm \ref{alg:linone} shows how one MinA can be computed in polynomial time. Since the loop in lines \ref{alg:linone:loop:begin}-\ref{alg:linone:loop:end} is executed exactly $n = |\tb|$ times, and the subsumption test also runs in time $\mathcal{O}(P(|\tb|))$, an overall time complexity polynomial in $|\tb|$ is achieved. The result is highlighted in \cite[Thm.~6]{orig1,orig2}.

\section{Implementations and Application in Practice}

As stated in the previous section, computing one MinA takes time polynomial in the number of axioms in the TBox, thus in the size of the knowledge base. While this might sound promising for practical applications and implementation, experimental reports show a different picture: In \cite[Sec.~5.2, p.~97]{meng-phd}, Suntisrivaraporn states that \enquote{The na\"{i}ve minimization algorithm did not terminate on \snomed in 48 hours}.

To make use of pinpointing in practice, \cite[Sec.~5]{orig1,orig2} proposes a trade-off. By relaxing the minimality property of computed axiom sets, significant gains in runtime can be realized. To decrease the size of the input for a black-box algorithm, a greedy algorithm based on Algorithm \ref{alg:allminas}, where the branch in lines \ref{alg:allminas:else:begin}-\ref{alg:allminas:else:end} is dropped, makes a pre-selection, purposefully breaking minimality. The remaining set is minimized using Algorithm \ref{alg:linone}.

With this approach, run times for the Galen \cite{galen} ontology decreased from 7 hours to approximately 10 minutes with an increase in size of 2.59\% \cite[Sec.~5]{orig1,orig2}.

Further empirical results are given in \cite[Sec.~6.2.5, p.~127]{meng-phd}.

\section{Summary}

We have presented a formalization of axiom pinpointing based on \cite{orig1,orig2} and covered a glass-box as well as two black-box algorithms to compute MinAs. While implementations for computing one MinAs are practicable, the complexity of enumerating all MinAs is not fully resolved.

\bibliography{ref}

\end{document}
